\documentclass[a4paper,12pt]{article}

\usepackage[utf8]{inputenc}
\usepackage{graphicx}
\usepackage{amsmath}

\usepackage{minted}

\begin{document}

\title{Configuration Parameters}
\author{Steen J. Sannung \texttt{sjs@logos.dk} \\
Logos Payment Solutions A/S }
\date{\today}

\maketitle

\section{}

\section{Lisa2 Definitions}

The Lisa2 parameter system is a client/server system, used for administering and deploying configuration parameters throughout our software solutions.
\begin{itemize}
\item Lisa2 Client is a software solution, which changes/reads the parameters, and puts them to use in realworld context. 
\item Lisa2 Server is a database which contains the values and ids for each "Lisa2 Client".
\end{itemize}

The PayManager application, implements a "Lisa2 Client", capable of changing/viewing parameter values for other "Lisa2 Clients". Much of
the design has been centered on making PayManager capable of displaying a sensible user interface. 
There is a configuration for each parameter, which details how it should be shown/edited in PayManager.

The following software definitions, defines a parameter set. Generally speaking a parameter is defined in the context of a Lisa2 Client - 
meaning there is a list of parameters id's mapped to values for each Lisa2 Client.

\begin{itemize}
\item Logos Id is an identifier which uniquely identifies a Lisa2 Client. 
\item Parameter Id is an identifier which uniquely identifies a parameter and it's configuration.
\item Parameter Configuration is a parameter specific list of attributes used for defining how values must be displayed/edited and transferred. The is only one configuration for each Lisa2 Server.
\item Parameter Value is a string values, which is related to Parameter Id. Parameter Values can be different for each Lisa2 Client.
\item Parameter Range is a Lisa2 Client specific list of values, defined for each parameter id. This values acts as a hint at how the Parameter should be displayed. 
When a range is set PayManager will display a "drop down list" with suggested values.
\end{itemize}

The Lisa2 parameter system uses a list of unique keys (parameter index), which are each mapped to a string parameter value.
The parameter list can be maintained for each terminal from PayManager.  

For each parameter there is a set of parameter configurations, which are meant to act as guides for displaying/editing values from PayManager. 
The parameter configuration has little value for the terminal, and is only used for optimizing which parameters are sent to the server.
q
Each terminal contains it's own list of parameter configurations, which is meant to be a copy of the server's parameter configuration. 
It is a topic of many discussions, what value the parameter configurations hold for the terminal
  

There are two values, which are helpful in "restricting" user input in PayManager:
\begin{itemize}
\item A terminal specific parameter range, which is stored for each parameter index. This range is formatted as a semicolon(;) delimited list of string identifiers. 
The list does not prohibit the input of "wrong" values in the specific parameter value, but it makes PayManager display a "drop down list" with suggested values.
\item Each parameter is born with a type parameter related to it. This type is a string describing the type. Possible values are "string", "dateTime", "integer", "headline",
"subheadline", "boolean", "date", "ipv4addr", "ipv4port". The type is a hint to PayManager on how to display/edit each parameter.
\end{itemize}

The model presented in this document, aims at introducing "use cases" for problems involiving the use of the Lisa2 parameter system.

\begin{itemize}
\item Modularity: It's essential to develop modules independently, avoiding reliance on global identifiers to simplify testing and maintenance.
\item Typed Values: Values within configurations must adhere to specific data types.
\subitem The model must incorporate use cases which encourages the use of library conversions between Lisa2 values and C++ variables. 
\item Dynamic Nature: Values may change over time, necessitating mechanisms to track these changes.
\end{itemize}


\section{Modularity}
The main problem about using global values to refer to specific parameters in Your code, is that the "identifier" is indeed a global
and can thus be viewed as a magic value, even though it is tied to a database of global identifiers.

There solution to this problem, it to 

\section{Typed values}

The most practical approach for a value model is to employ the . 
Nevertheless, a drawback of this model is its inability to perform compile-time checks to ensure that the type matches the parameter in question. 
Thinking about how
One potential solution to this issue involves implementing a templated approach. This would utilize C++'s own typing system to define values. 

\begin{minted}{c++}
template <typename T>
class IParameter {
public:
    virtual T get( const T& defaultValue ) const = 0;
    virtual void set( const T& value ) = 0;
};
\end{minted}

All parameters could be accessible through a Factory, using the following pattern:

\begin{minted}{c++}
class IParameterFactory {
public:
  IParameter<int> terminalId() {
    return new IParameterHelper<int>( TERMINAL_ID_UID );
  }

  
};
\end{minted}


\subsection{Reading parameters in a module}
Basically paramters are 


\section{What is a tag}

The tag is a universal identifier, which must be registered centrally, in order to be universal within the scope of the parameter.  

For each identifier, a number of values exist.

To communicate 

\section{An interface to send}

\end{document}
